\input{preamble}
\begin{document}

\title{Определение энергии $\alpha$-частиц по величине их пробега в воздухе}
\thanks{4.1}

\input{name}

\begin{abstract}
Измеряется пробег $\alpha$-частиц в воздухе двумя способами: с помощью торцевого счетчика Гейгера и сцинтилляционного счетчика, - по полученным величинам определяется энергия частиц.

\end{abstract}

\pacs{Valid PACS appear here}

\maketitle

\begin{enumerate}

\item 
\textbf{Исследование пробега $\alpha$-частиц с помощью счетчика Гейгера}\\
Включим установку и проверим ее функционирование. Дадим ей прогреться. Проведем предварительные измерения для определения вида зависимости количества частиц от расстояния до счетчика Гейгера.

Снимем зависимость скорости счета N за 100 секунда от расстояния между источником и счетчиком.

\begin{figure}[h]
\center{\includegraphics[scale=0.17]{my_plot1.png}}
\end{figure}

Построим график зависимости $N=N(x)$, и определим по нему средний и экстраполированный пробег $\alpha$-частиц
\begin{gather*}
R_{\text{э}} = (0.9 \pm 0.2 + 1.0)~cm = (2.1 \pm 0.2)\cdot10^{-3}~g/cm^2\\
R_{\text{ср}} = (0.7 \pm 0.2 + 1.0)~cm = (1.8 \pm 0.2)\cdot10^{-3}~g/cm^2.
\end{gather*}
\item
\textbf{Определение пробега $\alpha$-частиц с помощью сцинтилляционного счетчика}\\
Включим установку и проверим ее функционирование. Дадим ей прогреться. Проведем контрольные опыты и предварительные измерения.

Изменяя давление в камере, измерим количество частиц, фиксируемых счетчиком за 100 секунд.

\begin{figure}[h]
\center{\includegraphics[scale=0.17]{my_plot2.png}}
\end{figure}

Построим график зависимости $N=N(P)$, и определим по нему средний и экстраполированный пробег $\alpha$-частиц, а так же их энергию
\begin{gather*}
R_{\text{ср}}^{1, 1}((220 \pm 5)~torr,~300K) = 9~cm \Rightarrow\\
R_{\text{ср}}^{0, 0}(P_0, T_0) = \frac{T_0}{T_1} \frac{P_1}{P_0}R_{\text{ср}}^{1, 1} = (2.5 \pm 0.1)~cm =\\
= (3.0 \pm 0.1)\cdot10^{-3}~g/cm^2.\\
R_{\text{э}}^{2, 1}((310 \pm 5)~torr,~300K) = 9~cm \Rightarrow\\
R_{\text{э}}^{0, 0}(P_0, T_0) = \frac{T_0}{T_1} \frac{P_2}{P_0}R_{\text{ср}}^{2, 1} = (3.5 \pm 0.1)~cm =\\
= (4.3\pm0.1)\cdot10^{-3}~g/cm^2.\\
E_{\text{ср}} = \left( \frac{R_{\text{ср}}}{0.32} \right)^{2/3} = (3.9 \pm 0.2)~MeV, \\
E_{\text{э}} = \left( \frac{R_{\text{э}}}{0.32} \right)^{2/3} = (4.9 \pm 0.2)~MeV.
\end{gather*}
% Из сравнения результатов определим толщину слюды закрывающей окно торцевого счётчика.

% Считая, что эффективность счёта $\alpha$-частиц равна 100$\%$, оценим по известному периоду полураспада количество вещества в препарате. 

\item
\textbf{Исследование пробега $\alpha$-частиц с помощью ионизационной камеры}\\
Включим установку и проверим ее функционирование. Дадим ей прогреться. Проведем контрольные опыты и предварительные измерения. 
% Изменяя давление в камере, измерим ионизационный ток в камере.
Построим график зависимости $I=I(P)$, и определим по нему экстраполированный пробег $\alpha$-частиц, а так же их энергию

\begin{gather*}
R_{\text{э}}^{3, 1}((510 \pm 5)~torr,~300K) = 5~cm \Rightarrow\\
R_{\text{э}}^{0, 0}(P_0, T_0) = \frac{T_0}{T_1} \frac{P_3}{P_0}R_{\text{ср}}^{3, 1} = (3.2 \pm 0.1)~cm =\\
= (3.9\pm0.1)\cdot10^{-3}~g/cm^2.\\
E_{\text{э}} = \left( \frac{R_{\text{э}}}{0.32} \right)^{2/3} = (4.6 \pm 0.2)~MeV.
\end{gather*}

\begin{figure}[h]
\center{\includegraphics[scale=0.17]{my_plot3.png}}
\end{figure}

Сравним значение энергии с табличным
\begin{gather*}
E_{\text{table}} = 5.15~MeV.
\end{gather*}

\end{enumerate}

\end{document}