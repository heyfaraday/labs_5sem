\input{preamble}
\begin{document}

\title{Экспериментальная проверка уравнения Эйнштейна для фотоэекта}
\thanks{1.1}

\input{name}

\begin{abstract}
В работе исследуется зависимость фототока от величины задерживающего потенциала и частоты падающего излучения, что позволяет вычислить величину постоянной Планка.

\end{abstract}

\pacs{Valid PACS appear here}

\maketitle

\begin{enumerate}

\item 
\textbf{Градуировка монохроматора}\\
Используя окуляр, проградуируем барабан монохроматора по спектру неоновой лампы. Построим градуировочную кривую монохроматора.
\begin{figure}[h]
\center{\includegraphics[scale=0.18]{my_plot1.png}}
\end{figure}
\item
\textbf{Исследование зависимости фототока от величины запирающего потенциала}\\
На одном листе построим серию графиков $\sqrt{I} = f(V)$; для каждой длины волны определим величину запирающего потенциала, экстраполируя полученные кривые к оси абсцисс.
\begin{figure}[h]
\center{\includegraphics[scale=0.18]{my_plot2.png}}
\end{figure}
Построим график зависимости $V_0(\omega)$. По графику определим постоянную Планка, оценим погрешность результата и сравним найденное значение с табличным. 
\begin{figure}[h]
\center{\includegraphics[scale=0.18]{my_plot3.png}}
\end{figure}
\begin{gather*}
b = \frac{dV_0}{d\omega} = (5.9 \pm 0.7)~10^{-16}~V \cdot s,\\
\hbar = (0.9 \pm 0.1)~10^{-34}~J \cdot s,\\
\hbar = 1.054 ~10^{-34}~J \cdot s.
\end{gather*}
Оценим по графику красную границу в ангстремах и работу выхода материала катода в электронвольтах (с точностью до контактной разницы потенциалов).
\begin{gather*}
\lambda_{\text{красн}} = 150~nm,\\
W = 1.1~\text{eV}.
\end{gather*}

\clearpage


\begin{figure}
\center{\includegraphics[scale=0.16]{1700.png}}
\center{\includegraphics[scale=0.16]{1800.png}}
\center{\includegraphics[scale=0.16]{23.png}}
\end{figure}
\begin{figure}
\center{\includegraphics[scale=0.16]{22.png}}
\center{\includegraphics[scale=0.16]{20.png}}
\center{\includegraphics[scale=0.16]{15.png}}
\end{figure}
\begin{figure}
\center{\includegraphics[scale=0.16]{12.png}}
\center{\includegraphics[scale=0.16]{10.png}}
\center{\includegraphics[scale=0.16]{8.png}}
\end{figure}

\begin{figure}[h]
\center{\includegraphics[scale=0.35]{my_plot4.png}}
\end{figure}

\end{enumerate}

\end{document}