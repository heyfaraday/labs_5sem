\input{preamble}
\begin{document}

\title{Изучение спектров атома водорода и молекулы йода}
\thanks{2.2, 2.3}

\input{name}

\begin{abstract}
В работе исследуются: а) сериальные закономерности в оптическом спектре водорода; б) спектр поглощения паров йода в видимой области.

\end{abstract}

\pacs{Valid PACS appear here}

\maketitle

\begin{enumerate}

\item 
\textbf{Градуировка спектрометра}\\
Проградуируем спектрометр по спектру неона и ртути и построим градуировочную кривую.

\item
\textbf{Спектр водорода}\\
Измерим положение линий $H_{\alpha}$, $H_{\beta}$, $H_{\gamma}$ и $H_{\delta}$. Определим длины волн этих линий с помощью калибровочного графика.
\begin{gather*}
\lambda_{\alpha} = 6568~\text{\AA},\\
\lambda_{\beta} = 4872~\text{\AA},\\
\lambda_{\gamma} = 4346~\text{\AA},\\
\lambda_{\delta} = 4113~\text{\AA}.
\end{gather*}

Убедимся в том, что отношения длин волн водородных линий соответствуют формуле сериальной закономерности.
\begin{gather*}
n_{\alpha} = 2.99,\\
n_{\beta} = 3.98,\\
n_{\gamma} = 4.98,\\
n_{\delta} = 5.92.
\end{gather*}

Для каждой из наблюдаемых линий водорода вычислим значение постоянной Ридберга, определим её среднее значение по всем измерениям и оценим погрешность измерения. Сравните результаты опыта с расчётным значением $R$.
\begin{gather*}
R_{\alpha} = 109622~\text{cm}^{-1},\\
R_{\beta} = 109469~\text{cm}^{-1},\\
R_{\gamma} = 109569~\text{cm}^{-1},\\
R_{\beta} = 109409~\text{cm}^{-1}.\\
\\
R = (109517 \pm 55)~\text{cm}^{-1},\\
R_{\text{table}} = 109737.3~\text{cm}^{-1}
\end{gather*}

\item
\textbf{Спектр йода}\\
По градуировочной кривой монохроматора определим длины волн линий поглощения йода, соответствующие делениям барабана монохроматора $n_{1,0}$, $n_{1,5}$, $n_{\text{гр}}$.
\begin{gather*}
\lambda_{1,0} = 6369~\text{\AA},\\
\lambda_{1,5} = 5495~\text{\AA},\\
\lambda_{\text{гр}} = 5158~\text{\AA}.
\end{gather*}

Вычислим в электронвольтах энергию колебательного кванта возбуждённого состояния молекулы йода:
\begin{gather*}
h\nu_2 = (h\nu_{1,5} - h\nu_{1,0})/5 = 0.06~\text{эВ}.
\end{gather*}

Используя полученные результаты, а также данные о том, что энергия колебательного кванта основного состояния $h\nu_1 = 0,027~\text{эВ}$, а энергия возбуждения атома $E_A = 0,94~\text{эВ}$ вычислим:

а) энергию электронного перехода 
\begin{gather*}
h\nu_{\text{эл}} = (E_2 - E_1) = \frac{1}{2}(h\nu_1 - h\nu_2) = 1.65~\text{эВ},
\end{gather*}

б) энергию диссоциации молекулы в основном состоянии
\begin{gather*}
D_1 = h\nu_{\text{гр}} - E_A = 1.46~\text{эВ},
\end{gather*}

в) энергию диссоциации молекулы в возбуждённом состоянии \begin{gather*}
D_2 = h\nu_{\text{гр}} - h\nu_{\text{эл}} = 0.75~\text{эВ}.
\end{gather*}

\begin{figure}[h]
\center{\includegraphics[scale=0.40]{my_plot1.png}}
\end{figure}

\end{enumerate}

\end{document}